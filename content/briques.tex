Le présent chapitre étant le premier, je m'en vais commencer par le
commencement. En l'occurrence, l'algèbre Booléenne est au cœur de la
pensée mathématique. Les notions d'équivalence, d'implications, de
conjonction, etc. sont tous simplement incontournables aussi nous
n'allons pas essayer de les contourner. Une fois la logique booléenne
aquise, on corsera le tout en raisonnant sur des ensembles d'abord de
façon intuitive puis plus rigoureusement en utilisant des
quantificateurs. Ensuite, on utilisera ces magnifiques outils
conceptuels pour construire les entiers (le genre de truc qui est
vaguement utile puisqu'il se trouve absolument au \emph{centre} des
maths) et pour voir la notion de fonction (qui est à ranger juste à
côté des entiers).

\section{Algèbre Booléenne}


Nommée ainsi en l'honneur de M. Boole, mathématicien brittanique du
XIXème siècle, l'algèbre de Boole traite des relations logiques de
base qu'il peut exister entre différentes déclarations. Comme nous
allons le voir, on peut associer des variables à ces déclarations et,
en fait, ce sont ces variables que nous allons étudier. Les-dites
variables seront nommées ici $x,y$ ou encore $x_1,x_2,$ etc., comme
l'exige une tradition impliquant DesCartes et des matheux arabes.

\subsection{Opérations de base}


Pour passer des déclarations (ou affirmations) aux variables, il
suffit de procéder de la façon suivante.

Mettons que j'affirme que ``je porte un pantalon''. C'est mon droit le
plus strict et, si vous voulez mon avis, c'est plutôt mieux comme
ça. Créons la variable $x$ de telle façon que $x$ vaille 1 si ma
déclaration est vraie et 0 si elle fausse. Si vous démontrez que
$x=1$, alors vous savez que je porte un pantalon. Dans le cas
contraire, vous savez qu'il faut détourner poliment les yeux. Bien.


J'affirme maintenant que ``je porte un T-shirt''. Associons à cette
déclaration la variable $y$, de la même façon que précédemment. Comme
vous êtes une personne de bonnes mœurs, vous voulez vous assurer de ce
que je porte au moins un pantalon \emph{ou} un T-shirt. Et comme vous
êtes une personne pressée, vous ne voulez pas perdre de temps à
regarder deux variables pour le savoir, en l'occurrence $x$ et
$y$. Vous aimeriez bien en créer une qui vous donne directement
l'information voulue. En fait, vous voudriez une variable $z$ telle que:
\begin{itemize}
  \item Si $x=1$, alors $z=1$ (si je porte un pantalon, vous êtes content).
  \item Si $y=1$, alors $z=1$ (si je porte un T-shirt, vous êtes content).
  \item Si $x=0$ et $y=0$, alors $z=0$ (si je ne porte ni pantalon ni
    T-shirt, vous n'êtes \emph{pas} content).
\end{itemize}

Ceci peut se réécrire d'une façon un peu plus redondante mais plus
facilement généralisable en utilisant une table de vérité. On appelle
ainsi un tableau qui regroupe toutes les valeurs possibles des
variables considérées (ici, $x$ et $y$) et qui donne pour chaque
combinaison la valeur d'une autre variable (ici, $z$). La figure
\ref{fig:table-ou} donne celle qui nous intéresse.

\begin{figure}[ht]
  \centering
  \begin{tabular}{ll|l}
    x & y & z \\ \hline
    0 & 0 & 0 \\
    1 & 0 & 1 \\
    0 & 1 & 1 \\
    1 & 1 & 1 \\
  \end{tabular}
  \caption{Votre première table de vérité! Que d'émotions!}
  \label{fig:table-ou}
\end{figure}

Vous noterez qu'en fait, $z$ est vrai (c'est-à-dire ``vaut'') si $x$
\emph{ou} $y$ le sont. Du coup, on peut écrire tout simplement que
``$z = x ~OU~ y$''. Comme on est décidément très pressés, on peut
écrire ça de façon plus compact: $z = x \lor y$. La table de vérité
figure \ref{fig:table-ou} est donc tout simplement celle du
\introduceterm{OU} logique.

Une personne extrêmement à cheval sur le code vestimentaire serait
cependant rassurée de savoir que je porte un pantalon
\introduceterm{ET} un T-shirt. Dans ce cas, elle peut se créer une
nouvelle variable $u$ obéissant à la table de vérité figure
\ref{fig:table-et}.

\begin{figure}[ht]
  \centering
  \begin{tabular}{ll|l}
    x & y & u \\ \hline
    0 & 0 & 0 \\
    1 & 0 & 0 \\
    0 & 1 & 0 \\
    1 & 1 & 1 \\
  \end{tabular}
  \caption{La table de vérité du ET logique}
  \label{fig:table-et}
\end{figure}

Fort logiquement,\footnote{Ce qui est plutôt une bonne chose dans la
  mesure ou le présent chapitre prétend vous inculquer les bases de la
  logique formelle} on peut écrire que $u = x~ ET~ y$. Là encore, on
peut l'abréger en $u = x \land y$.

Imaginons maintenant que je possède une chemise hawaïenne.\footnote{Ce
  n'est bien sûr pas le cas, j'ai une dignité quand même. C'est juste
  pour l'exemple.} Créons encore une nouvelle variable $h$ qui soit
comme précédemment vraie (c'est-à-dire égale à 1) si je la porte et
fausse dans le cas contraire (donc égale à 0). Dans ce cas, si on
associe une variable $v$ à votre bonheur d'esthète, il est nécessaire
que $h$ et $v$ aient des valeurs opposées: si l'un est vrai, l'autre
doit être faux (et vice-versa). La table de vérité de la relation que
vous voulez est donnée figure \ref{fig:table-non}.

\begin{figure}[ht]
  \centering
  \begin{tabular}{l|l}
    x & v \\ \hline
    0 & 1 \\
    1 & 0 \\
  \end{tabular}
  \caption{La table de vérité du NON logique}
  \label{fig:table-non}
\end{figure}

Comme toujours, il existe une façon croquignolette d'exprimer l'idée
que $v$ a la valeur opposée de celle de $h$. On dit que $v$ est la
\introduceterm{négation} de $h$ et on le note $v = \neg h$.

\plainbreak{2}

Vous connaissez maintenant toutes les opérations logiques de base,
vous êtes donc prêt à construire un ordinateur et à démontrer
l'incomplétude de Gödel. Les opérations en question sont:

\begin{description}
  \item[Disjonction]
    C'est à dire prendre le OU de deux
    variables. On la note avec le symbole $\lor$.
  \item[Conjonction]
    C'est à dire prendre le ET de deux
    variables. On la note avec le symbole $\land$.
  \item[Négation]
    C'est à dire prendre la valeur opposée d'une
    variable. On la note avec le symbole $\neg$.
\end{description}

On appelle \introduceterm{formule} un ensemble de variables logiques
liées par ces relations, par exemple $(x \land \neg y) \lor z$ ou bien
$\neg x \lor ((y \land z) \lor u)$

\subsection{Creusons un peu plus}


\subsection{Un exemple: résoudre un sudoku}

\subsubsection{Problème}

J'imagine que vous savez tous ce qu'est un sudoku mais, dans le cas
ou certains de mes lecteurs bien-aimés l'ignoreraient, laissez-moi
vous en rappeler le principe. On dispose d'une grille de 9 cases par 9
cases et on doit la remplir avec des chiffres (entre 1 et 9) de telle
façon que:
\begin{itemize}
  \item Le même chiffre n'apparaisse pas deux fois sur une même ligne.
  \item Le même chiffre n'apparaisse pas deux fois sur une même
    colonne.
  \item Le même chiffre n'apparaisse pas deux fois dans l'un des 9
    sous-carrés de taille 3x3.
\end{itemize}
Une instance de sudoku correspond à une telle grille partiellement
remplie. Le but du jeu est de finir de la remplir en respectant les
règles sus-mentionnées. Qu'est-ce que ça fait classe de placer
``sus-mentionné'' dans un texte, ça fait tout de suite très sérieux.

En bon matheux, résoudre \emph{une} grille \emph{particulière} ne nous
intéresse pas. Ce qu'on veut, c'est résoudre le cas général; c'est à
dire trouver une stratégie efficace pour résoudre rapidement n'importe
quelle instance. À la condition d'avoir le programme qui va bien, on
peut le faire en utilisant des formules logiques.


\subsubsection{Stratégie}

Vu que j'ai cassé le suspens en disant qu'on allait le faire en
utilisant des formules logique, je ne peux pas vous demander de
trouver cette idée. Surtout que bon, vu que c'est dans une
sous-section ``exemple'' d'une partie sur l'algèbre de Boole, vous
pouviez vous douter qu'on allait s'en servir. Mais je m'égare.

Pour pouvoir écrire une formule logique, il nous faut des
variables. Ça me semble être un début raisonnable, j'espère que vous
serez d'accord. J'impose unilatéralement l'ensemble de variables
suivant: les $x_{i,j,k}$.

Il y a trois indices\footnote{C'est les petites lettres après le nom
  générique de la variable, en l'occurrence $i$, $j$ et $k$.} qui
correspondent à trois données. $i$ est, comme ce sera très souvent le
cas ici, l'indice de la ligne d'une case. $j$ est celui de sa
colonne. Jusque là ça va (j'espère). Maintenant, qu'est-ce que c'est
que ce $k$? En fait, on va associer à chaque case 9 variables
correspondant aux neuf chiffres pouvant y résider. $k$ correspond à
ces 9 chiffres.

Ces variables sont telles que si il y a le chiffre $k$ dans la case de
la ligne $i$ qui est dans la colonne $j$, alors $x_{i,j,k}$ est
vrai; $i,j,k$ pouvant varier entre 1 et 9. Par exemple, si $x_{1,1,1}$
est vrai, alors il y a un 1 dans la première case de la première
ligne. Si $x_{9,9,5}$ est vrai, alors il y a 5 dans la case tout en
bas à droite. Si $x_{5,5,1}$ est faux alors il n'y a pas de 1 dans la
case en plein milieu de la grille. Vous suivez?

Mais pourquoi est-ce qu'on s'embête avec ces variables me
demanderez-vous? Pour en faire une jolie formule vous répondrai-je! En
effet, les différentes contraintes évoquées précédemment peuvent se
traduire en formules logique utilisant des ET, des OU et des NON.

\begin{itemize}
  \item \emph{Pas deux fois le même chiffre dans une même ligne:}
    Cette condition peut être réécrite en disant qu'il est impossible
    d'avoir $x_{i,j,k}$ et $x_{i,j',k}$ vrai en même temps; c'est à
    dire que sur la ligne $i$; les cases dans les colonnes $j$ et $j'$
    ne peuvent pas toutes les deux avoir la valeur $k$. En d'autres
    termes, pour toutes les lignes d'indice $i$ allant de 1 à 9:
    \begin{equation}
      \neg ( x_{i,j,k} \land x_{i,j',k})
    \end{equation}
  \item \emph{Pas deux fois le même chiffre dans une même colonne:}
    Cett
\end{itemize}
